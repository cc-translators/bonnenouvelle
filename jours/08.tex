\jrnlday{Au parfait moment}

\dvepigraph{%
Et toi, Bethléhem Éphrata, toi qui es petite parmi les milliers de Juda, de toi sortira pour moi, Celui qui dominera sur Israël et dont l’origine remonte au lointain passé, aux jours d’éternité.}{\ibibleverse{Mi}(5:2)}

Jésus est né pendant le règne du plus grand empereur romain, César Auguste. Cet empereur païen, considéré par beaucoup comme un dieu, régnait sur l'Empire Romain quand Dieu s'est fait un homme, et c'est César Auguste, qui ignorait tout des prophéties messianiques, que Dieu a pourtant utilisé pour faire que Jésus naisse à Bethléhem tout comme le prophète Michée l'avait prédit.

À cette époque de l'histoire, la domination de Rome sur le reste du monde était à son apogée, ce qui faisait de César Auguste plus ou moins l'empereur du monde entier. Le général romain Pompée avait conquis Jérusalem 60~ans plus tôt et de ce fait les Juifs étaient aussi des citoyens romains, et pour la première fois, on leur demandait de se faire recenser pour payer des impôts. César Auguste, un empereur qui taxait et dépensait beaucoup, publia un édit impérial ordonnant à chacun de retourner dans la ville de ses ancêtres et de s'y faire recenser par l'administration des impôts.

En raison de ce décret, Marie et Joseph se rendirent à Bethléhem, la ville de leurs ancêtres. Ils venaient à peine d'arriver en ville que l'accouchement de Marie se déclencha et, rapidement, le bébé Jésus est né, tout juste comme le prophète Michée l'avait prédit 750~ans auparavant.

Même dans les situations d'oppression, Dieu travaille pour notre bien. Comme pour Marie, Joseph et Jésus, Dieu sait exactement ce qu'Il fait dans nos vies. Peut-être vous retrouvez-vous dans une situation désagréable qui dérange vos habitudes de façon radicale. Cheminer à dos d'âne de Nazareth à Bethléhem en passant par les montagnes de Palestine n'était pas vraiment une partie de plaisir pour Joseph et \ocadr surtout! \fcadr{} Marie, mais Dieu avait Sa bonne main sur eux. Ils sont sortis de ce voyage à Bethléhem totalement sains et saufs parce que c'était le plan de Dieu qui s'accomplissait.

De la même façon, Dieu travaille dans les circonstances de nos vies aujourd'hui. Nous ne comprenons peut-être pas ce qu'Il est en train de faire, mais nous pouvons Lui faire confiance et nous reposer dans Son amour et dans le fait qu'Il prend soin de nous. Son plan et son calendrier sont parfaits.

\begin{dvquotes}
\dvquote{%
Nous savons, du reste, que toutes choses coopèrent au bien de ceux qui aiment Dieu, de ceux qui sont appelés selon son dessein.
}{\ibibleverse{Rm}(8:28)}
\end{dvquotes}

