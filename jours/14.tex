\jrnlday{L'auberge de Kimham}

\begin{savenotes}% to add the footnote to the end of the page
\dvepigraph{%
Et elle enfanta son fils premier-né. Elle l’emmaillota et le coucha dans une crèche, parce qu’il n’y avait pas de place pour eux dans l’auberge\footnote{\og l'hôtellerie \fg{}, dans la version \og à la Colombe \fg{}}}{\ibibleverse{Lc}(2:7)}
\end{savenotes}

L'auberge où Joseph et Marie ont cherché de la place pourrait très bien avoir été un endroit avec une longue tradition biblique d'hospitalité.

Mille ans avant le Christ, quand David était roi d'Israël, son fils Absalon a conspiré contre lui et lui a volé le trône. David s'est alors enfui de Jérusalem pour se cacher en Galaad, à 100 kilomètres de là, à l'Est du Jourdain. Pendant qu'il était là, un riche propriétaire terrien du nom de Barzillaï le Galaadite, est devenu son ami, l'a reçu chez lui et a pourvu à tous ses besoins ainsi qu'à ceux de son entourage.

Quand David est remonté sur le trône, il a invité Barzillaï à venir en Israël avec lui, mais Barzillaï qui avait quatre-vingts ans, aimait bien sa maison, son environnement familier "et un peu de paix et de tranquillité, si vous permettez". Aussi Barzillaï a t-il demandé à David "Pourquoi ne prendrais-tu mon fils Kimham à ma place?" David a accepté, et dès lors, Kimham est devenu un membre du cercle intime de ses amis loyaux (2 Samuel 19:31-40).

Des années plus tard, Kimham a bâti un gîte, une maison près de Bethléhem pour y accueillir les voyageurs et l'a appelée "Geruth Kimham" ce qui peur être traduit par "Auberge de Kimham". Il est évident que l'esprit d'hospitalité que Barzillaï avait pour David, demeurait aussi en Kimham.

Avançons rapidement dans le temps jusqu'à l'époque du prophète Jérémie, un temps de grand tumulte et de chaos – Juda était conquise et détruite par Babylone, Jérémie était kidnappé et forcé d'aller en Égypte. En chemin, ils ont eu une courte période de repos quand ils se sont arrêtés dans un gîte près de Bethléhem Vous l'avez deviné... c'était l'auberge de Kimham. Ce fut l'un des rares moments de repos dans la difficile vie de Jérémie.

En Hébreu, "Kimham" signifie "désir ardent et douloureux". Il est très possible que ce soit précisément l'auberge où Joseph et Marie sont arrivés quelques six cents ans plus tard. L'auberge de Kimham, un endroit connu pour l'hospitalité, le refuge et le repos offert aux voyageurs traversant des circonstances difficiles.

C'était la situation de Marie, qui venait de voyager sur des routes de montagne au neuvième mois de sa grossesse et qui aurait très bien pu se trouver en phase de début d'accouchement.

Quand ils sont arrivés à l'auberge, Joseph a dû supplier l'aubergiste de leur donner une chambre, malgré le clignotement du signe "Complet" en néon rouge qui aveuglait ses yeux! Peut-être l'aubergiste était-il un descendant de Kimham et Barzillaï. Mais ce n'est parce qu'il venait d'une famille qui pratiquait l'hospitalité qu'il la pratiquait nécessairement aussi.

Vous vous dites peut-être "comment aurait-il pu leur donner une chambre s'il n'en restait plus une seule?" Mais en fait, il en restait une... la chambre de l'aubergiste! Comme ça aurait été merveilleux, s'il avait donné sa propre chambre à Joseph et Marie pour la naissance de Jésus! La maison de l'aubergiste aurait été le lieu de naissance du Sauveur et il aurait été le premier à voir l'enfant Christ.

Il se trouve que cet honneur a été donné aux bergers parce que l'aubergiste que le seul endroit qui lui restait était l'étable où il gardait ses moutons. L'aubergiste avait sommeil et ne voulait pas être dérangé et il est ainsi passé à côté du plus grand évènement de l'Histoire. Il l'a raté parce qu'il n'avait pas de place pour le Seigneur de gloire.

Qu'en est-il de vous? Jésus ne veut pas de n'importe quelle place. Il veut trouver Sa place dans votre cœur et votre vie, pour y demeurer et les remplir de Sa Joie, Son amour et Sa gloire. Avez-vous de la place pour Lui? Tous ceux qui accueillent Jésus dans leur vie ne le regrettent jamais.


\dvquote{%
Le mystère caché de tout temps et à toutes les générations, mais dévoilé maintenant à ses saints, à qui Dieu a voulu faire connaître quelle est la glorieuse richesse de ce mystère parmi les païens, c’est-à-dire: Christ en vous, l’espérance de la gloire.}{\ibibleverse{Col}(1:26,27)}

\dvquote{%
Voici: je me tiens à la porte et je frappe. Si quelqu’un entend ma voix et ouvre la porte, j’entrerai chez lui, je souperai avec lui et lui avec moi.}{\ibibleverse{Ap}(3:20)}

