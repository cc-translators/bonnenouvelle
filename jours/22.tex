\jrnlday{Les sages le recherchent}

\dvepigraph[2]{%
Jésus était né à Bethléhem en Judée, au temps du roi Hérode. Des mages\NdT{des \Og sages \Fg{}} d’Orient arrivèrent à Jérusalem et dirent\frcolon{}
\Og Où est le roi des Juifs qui vient de naître? Car nous avons vu son étoile en Orient, et nous sommes venus l’adorer \Fg{}
[\dots{}] Après avoir entendu le roi, ils partirent. Et voici\frcolon{} l’étoile qu’ils avaient vue en Orient les précédait; arrivée au-dessus (du lieu) où était le petit enfant, elle s’arrêta. À la vue de l’étoile, ils éprouvèrent une très grande joie.}{\ibibleverse{Mt}(2:1,2,9,10)}

Mille cinq cents ans avant la naissance de Christ vivait un prophète vraiment peu fiable appelé Balaam. En fait, Balaam était si inconséquent qu'à une occasion, le Seigneur a dû utiliser un âne pour le reprendre et lui donner des instructions parce qu'il désobéissait à Dieu.

Le nom de \emph{Balaam} signifie littéralement \Og échec \Fg{} et ça constituerait une description assez juste du personnage s'il n'\'etait pas l'auteur d'une grande prophétie au sujet du Messie\frcolon{} \Og Je le vois, mais non maintenant, je le contemple, mais non de près. Un astre sort de Jacob, un sceptre s’élève d’Israël \Fg{} (\ibibleverse{Nb}(24:17)).

Mille cinq cents ans plus tard, un groupe d'hommes sages (\Og les mages \Fg{}) venant de l'Orient ont pris leurs bagages et accompli un voyage de près de deux mille kilomètres. Ils étaient sûrs que le Roi des Juifs, le Messie, était né.

Peut-être que ces hommes qui contemplaient les étoiles avaient entendu parler de la prophétie de Balaam et recherchaient un signe de la part de Dieu. Une sorte d'étoile brillante dans le ciel leur a montré le chemin, conduisant les mages en Israël.

Ces hommes sages étaient enthousiasmés à l'idée de trouver le Roi nouveau-né. Pourtant, les prêtres et les scribes qui connaissaient les Écritures parlant du Messie n'étaient pas du tout émus par Sa venue. Les Mages avaient fait deux mille kilomètres pour voir Jésus et être près de Lui, mais les gens et les prêtres de Jérusalem ne voulaient même pas parcourir dix kilomètres pour venir trouver et adorer le Messie.

Ils étaient indifférents, bien à l'aise dans leur style de vie confortable et ne voulaient surtout pas être dérangés. Ce sont les sages qui ont soif de Christ, qui n'hésitent pas à voyager loin ou à endurer des difficultés et affronter des périls pour trouver et adorer le Messie.

Et vous? Voulez-vous rechercher davantage Dieu et sa Parole? De toute évidence, l'étoile a disparu à Jérusalem pendant un moment, mais les Mages ont continué à chercher, tout en étant, pour ainsi dire, dans le noir. Dieu les a dirigés et encouragés par Sa Parole, et alors qu'ils persistaient à croire et à suivre les prophéties, l'étoile est revenue, confirmant le chemin jusqu'à Christ.

Continuez à chercher et à suivre Jésus, même lorsque les circonstances sont contraires et que les sentiments sont absents. Croyez et soyez encouragés par la Parole, et très vite la présence de Christ sera de nouveau confirmée dans votre cœur.

% This is ugly...
%\enlargethispage{5\baselineskip}

\begin{dvquotes}
\dvquote{%
        Vous me chercherez et vous me trouverez, car vous me chercherez de tout votre c\oe{}ur.}{\ibibleverse{Jr}(29:13)}
\end{dvquotes}
