\jrnlday{Des mères bénies}


\dvepigraph{
En ces jours-là, Marie se leva et s’empressa d’aller vers les montagnes dans une ville de Juda. Elle entra dans la maison de Zacharie et salua Élisabeth. Dès qu’Élisabeth entendit la salutation de Marie, son enfant tressaillit dans son sein. Élisabeth fut remplie \mbox{d’Esprit~Saint}.}{\ibibleverse{Lc}(1:39-41)}

Habituellement, une mère peut sentir son bébé bouger à partir de trois mois et demi de grossesse, mais ce qui s'est passé ici était quelque chose de rare! Jean dansait selon l'Esprit dans le sein d'Élisabeth! Je crois que c'est l'accomplissement de la prophétie de Gabriel faite à Zacharie≡ \og Il sera rempli de l’Esprit Saint dès le sein de sa mère \fg{} (\ibibleverse{Lc}(1:15)).

Jean et sa mère Élisabeth furent tous deux remplis de l'Esprit Saint quand Jésus, à l'intérieur du sein de Marie, est entré dans la maison. En déversant Sa faveur sur Élisabeth et Marie de façon aussi surprenante et profuse, Dieu les a conduites à le connaître Lui et Son amour de façon plus profonde, et elles ont répondu par une louange pure et spontanée.

Élisabeth et Marie étaient de belles femmes prêtes à donner la vie et dont la maternité allait changer le monde entier et inaugurer une ère de salut, de libération et de grâce. Dieu montre qu'il a la plus haute estime pour le rôle de mère par la façon dont ils traitent ces deux femmes. Alors que pour notre société laïque, avoir des enfants et s'occuper d'eux est considéré comme une chose insignifiante, ennuyeuse, comme un gaspillage de talents, pour Dieu, être mère est une vocation très noble!

Mamans, réjouissez-vous de votre vocation! En élevant vos enfants dans les voies du Seigneur, qui peut savoir ce qu'il en résultera si vous êtes fidèles à Dieu? Chaque enfant représente un potentiel formidable, même ceux qui se comportent de façon impossible. Dieu a un plan pour nos enfants et ceux qu'ils affecteront, et les mères constituent une partie inestimable de ce plan. Aimez vos enfants et éduquer les et voyez Dieu à l'œuvre!

\dvquote{%
Oriente le jeune garçon sur la voie qu’il doit suivre; Même quand il sera vieux, il ne s’en écartera pas.
}{\ibibleverse{Pr}(22:6)}

