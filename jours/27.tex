\jrnlday{Le Fils unique venant du Père}

\dvepigraph[2]{%
Au commencement était la Parole, et la Parole était avec Dieu, et la Parole était Dieu […] La Parole a été faite chair, et elle a habité parmi nous, pleine de grâce et de vérité; et nous avons contemplé sa gloire, une gloire comme celle du Fils unique venant du Père... Personne n'a jamais vu Dieu; Dieu (le Fils) unique, qui est dans le sein du Père, Lui, l'a fait connaître.}{\ibibleverse{Jn}(1:1,14,18)}

Chacun des quatre évangélistes relate la venue de Jésus dans le monde. Les récits de Matthieu et Luc parlant d'avant la naissance, de la naissance et de la petite enfance de Christ nous sont les plus familiers, mais Jean nous fait remonter dans le temps plus loin que les autres, et nous présente l'histoire de Noël du point de vue divin.

Le principal personnage du récit de Jean est \Og la Parole \Fg{}, l'idée centrale est que \Og la Parole a été faite chair \Fg{}, et le résultat est que Dieu s'est manifesté et s'est fait connaître Lui-même à l'homme.

Considérons de plus près ce personnage principal, la Parole. Le petit Jésus dont nous célébrons la naissance n'était pas un enfant ordinaire \ocadr Il était la Parole Éternelle qui était Dieu. Au commencement, la Parole existait déjà, et ici nous voyons l'existence éternelle de Jésus-Christ qui est la Parole.

Pour cette raison, Jésus est entièrement capable de répondre aux questions les plus profondes de l'homme sur l'éternité et l'existence\frcolon{} D'où est-ce que je viens? Pourquoi suis-je ici? Où est-ce que je vais? L'Éternelle Parole de Dieu ne donne pas simplement des réponses à ces questions, Il est la réponse.

\Og La Parole était avec Dieu \Fg{} ne veut pas dire qu'ils habitaient séparément. Au contraire, ils sont dans le sein du Père \ocadr unis dans l'amour, engagés dans une relation intime et en communion l'un avec l'autre. Cette caractéristique de Jésus Lui permet de répondre aux besoins les plus profonds des hommes. Que nous le réalisions ou non, ce que nous désirons le plus c'est l'amour et la communion de Dieu.

Que Dieu porte le titre de \Og la Parole \Fg{} indique qu'Il aime communiquer avec nous. Jésus et le Père ont communiqué intimement l'un avec l'autre depuis l'éternité passée. Dieu souhaitait aussi cela avec l'homme, mais la communication a été interrompue à cause du péché. Aussi Dieu s'est-Il mis à restaurer la communication en envoyant Son Fils Jésus pour se manifester et se faire connaître Lui-même auprès de nous.

\Og La Parole a été faite chair… \Fg{} Jusqu'à quelle extrémité Jésus n'est-il pas allé pour nous révéler Dieu! La barrière du péché \ocadr la désobéissance à la Parole de Dieu \fcadr{} a été enlevée et lavée par la mort de Jésus, ce qui a ouvert pour nous la voie de l'amitié et de la communion avec Dieu.

Confronté de nouveau à la Parole de Dieu aujourd'hui, détournez-vous de l'apathie et de la rébellion, recevez le Don de Dieu, et entrez en communion avec Lui pour toujours.

% This is ugly...
\enlargethispage{5\baselineskip}

\dvquote{%
Ayez en vous la pensée qui était en Christ-Jésus, Lui dont la condition était celle de Dieu, Il n'a pas estimé comme une proie à arracher d'être égal avec Dieu, mais Il s'est dépouillé Lui-même, en prenant la condition d'esclave, en devenant semblable aux hommes.}{\ibibleverse{Ph}(2:5-7)}


