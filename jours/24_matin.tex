\jrnlday[post={, matin}]{Le Miracle de Bethléhem}

\dvepigraph{%
Et toi, Bethléhem Éphrata toi qui es petite parmi les milliers de Juda, de toi sortira pour moi Celui qui dominera sur Israël et dont l’origine remonte au lointain passé, aux jours d’éternité.}{\ibibleverse{Mi}(5:2)}

Le cantique de Noël \emph{Ô Petite Ville de Bethléhem} a été écrit par Phillip Brooks, le pasteur de l'Église de la Sainte Trinité à Philadelphie, pendant la Guerre Civile.

En décembre~1868, l'église avait besoin d'un chant pour le culte de Noël des enfants. Trois années plus tôt, le pasteur Brooks avait fait un voyage en Terre Sainte et avait assisté à un culte le soir de Noël à Bethléhem. Alors qu'il se remémorait cette soirée et réfléchissait au Saint Enfant qui était né à cet endroit même, Dieu lui donna le cantique, \emph{Ô Petite Ville de Bethléhem}. La chorale de l'église présenta ce chant et les enfants furent émerveillés. Le pasteur Brooks décéda en~1893. Plusieurs semaines après son enterrement, une fillette de cinq ans qui aimait beaucoup le pasteur s'inquiétait de ne plus le voir et demanda à sa mère\frcolon{} \Og Où est le pasteur Brooks? \Fg{}. Sa mère lui répondit\frcolon{} \Og Il est parti au ciel, mon chou. \Fg{} Le visage de la fillette s'illumina et elle s'exclama\frcolon{} \Og Oh, les anges doivent être si contents! \Fg{}

Bethléhem était un lieu de naissance approprié pour Jésus parce que cet endroit nous révèle les voies de notre Seigneur. C'était une toute petite ville à l'époque de la naissance de Christ qui ne devait pas compter plus de 500~habitants et qui était insignifiante au milieu des milliers de cités de Juda.

De toute évidence, il y avait d'autres villes appelées Bethléhem, et on distinguait celle-ci en l'appelant Bethléhem Éphrata. Nous n'aurions jamais entendu parler de Bethléhem si Dieu ne l'avait pas rendue célèbre.

Si j'étais Dieu et que j'envoyais mon Fils pour sauver l'humanité, je l'enverrais dans le Palais de César à Rome, mais jamais dans une étable à Bethléhem. C'est pourtant ce que Dieu a choisi, parce que c'est la façon d'agir de Dieu. Dieu n'est pas impressionné par l'orgueil, la grandeur, la célébrité des hommes. Jésus a dit\frcolon{}
\Og Venez à moi, vous tous qui êtes fatigués et chargés, et je vous donnerai du repos. Prenez mon joug sur vous et recevez mes instructions, car je suis doux et humble de coeur, et vous trouverez du repos pour vos âmes \Fg{} (\ibibleverse{Mt}(11:28-29)).

C'est la façon d'être de Dieu; Il s'approche des petits et des humbles. C'est très humiliant d'admettre que vous êtes un pécheur qui a tout gâché et qui n'a pas su rester à l'intérieur des limites établies par Dieu. Nous avons besoin d'un Rédempteur qui est né dans une étable, qui est mort sur une croix, et qui est sorti du tombeau.

Dieu choisit les faibles et les fous parce que c'est la voie du Seigneur. À l'époque de la naissance de Christ, les étables pour les vaches, les ânes, les moutons étaient souvent dans des grottes. À Bethléhem, il y a une petite cathédrale appelée l'Église de la Nativité et derrière l'autel de cette église se trouve une petite grotte que l'on croit être l'étable même où Jésus est né.

Quand vous entrez dans la grotte, vous devez vous baisser parce que le passage d'entrée est très bas, ce qui a conduit l'auteur chrétien américain Max Lucado à écrire\frcolon{} \Og Vous pouvez voir le monde en vous tenant debout, mais pour voir le Sauveur, vous devez vous mettre à genoux. Vous devez vous humilier pour admettre que vous avez des besoins, que vous n'êtes pas parfait ni autosuffisant. \Fg{} \Og Dieu résiste aux orgueilleux, mais il donne sa grâce aux humbles \Fg{} (\ibibleverse{IP}(5:5b)). C'est le miracle de Bethléhem.

\begin{dvquotes}
\dvquote{%
Humiliez-vous donc sous la puissante main de Dieu, afin qu’il vous élève en temps voulu. Déchargez-vous sur lui de tous vos soucis, car il prend soin de vous.}{\ibibleverse{IP}(5:6-7)}


\ornrule
\begin{verse}
\begin{altverse}
Petite ville de Bethléhem\\
tu dors tranquillement\\
Sur ton sommeil,\\
l'étoile d'or se lève au firmament.\\
Sa lumière éternelle\\
nous apporte la joie.\\
Oui, la réponse à nos appels,\\
ce soir se trouve en toi.
\end{altverse}
\end{verse}
\end{dvquotes}
