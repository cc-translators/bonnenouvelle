\jrnlday{Le cantique de Marie}


\dvepigraph[2]{%
Élisabeth fut remplie d’Esprit Saint et s’écria d’une voix forte\frcolon{} Tu es bénie entre les femmes, et le fruit de ton sein est béni. Comment m’est-il accordé que la mère de mon Seigneur vienne chez moi? Car voici\frcolon{} aussitôt que la voix de ta salutation a frappé mes oreilles, l’enfant a tressailli d’allégresse dans mon sein. Heureuse celle qui a cru à l’accomplissement de ce qui lui a été dit de la part du Seigneur.

Et Marie dit\frcolon{} Mon âme exalte le Seigneur et mon esprit a de l’allégresse en Dieu, mon Sauveur, parce qu’il a jeté les yeux sur la bassesse de sa servante.
Car voici\frcolon{} désormais toutes les générations me diront bienheureuse.
Parce que le Tout-Puissant a fait pour moi de grandes choses. Son nom est saint, et sa miséricorde s’étend d’âge en âge sur ceux qui le craignent Il a déployé la force de son bras; Il a dispersé ceux qui avaient dans le coeur des pensées orgueilleuses, Il a fait descendre les puissants de leurs trônes, élevé les humbles, rassasié de biens les affamés, renvoyé à vide les riches. Il a secouru Israël, son serviteur, et s’est souvenu de sa miséricorde, \ocadr comme il l’avait dit à nos pères \fcadr{}, envers Abraham et sa descendance pour toujours.
}{\ibibleverse{Lc}(1:41-55)}

Jésus est né à une époque très sombre et désespérée de l'histoire humaine. Le Seigneur n'avait plus parlé directement à l'homme depuis 400~ans. Après qu'Élisabeth a béni Marie d'une façon inouïe, Marie a été frappée par le plein impact de la faveur divine, et un merveilleux psaume est sorti de ses lèvres, comparable aux chants de prière de l'Ancien Testament. Marie était non seulement inspirée par l'Esprit Saint, mais elle connaissait aussi très bien les Écritures; Son cantique inclut plusieurs citations directes ou des références indirectes. Marie faisait confiance aux promesses de Dieu, et Dieu bénit ceux qui le font!

Le Cantique de Marie est aussi appelé le \emph{Magnificat} d'après le premier mot de sa traduction en latin. L'âme et l'esprit de Marie exaltaient, ou \Og magnifiaient \Fg{} le Seigneur en réponse à Dieu et à ses bénédictions, dans une pure louange et non pour obtenir quelque chose de Lui. Marie reçoit Sa faveur et ce faisant, Il devient plus réel et plus glorieux à ses yeux.

Le Cantique de Marie ne parle que du Seigneur, faisant dix-sept fois référence à Lui de façons diverses. Elle rend hommage à Ses attributs divins, exprimant d'abord le fait qu'Il est son Sauveur. Elle reconnaît son besoin de Lui, et dans sa belle humilité, elle voit Dieu~chasser et détruire son orgueil. Comme Marie, plus nous voyons Dieu clairement, plus nous reconnaissons de façon aiguë notre péché et notre besoin de Dieu.

Puis, Marie vante la sainteté du Seigneur, Sa pureté absolue et Sa perfection. Il démontre la puissance de Son nom en amenant la justice. Bien qu'étant tout-puissant, Il est plein de miséricorde \Og d'âge en âge pour ceux qui le craignent. \Fg{}

Notre Dieu est un révolutionnaire quand il s'agit de juger les systèmes sociaux cupides de notre monde, et Il est fidèle pour se souvenir de Ses promesses et pour toujours les tenir.

% This is ugly...
\enlargethispage{5\baselineskip}

\dvquote{Il a rassasié de biens les affamés, renvoyé à vide les riches. Il a secouru Israël, son serviteur, et s’est souvenu de sa miséricorde,  \ocadr comme il l’avait dit à nos pères, envers Abraham et sa descendance pour toujours.}{\ibibleverse{Lc}(1:53-55)}

