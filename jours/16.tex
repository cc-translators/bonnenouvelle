\jrnlday{De bons bergers}

\dvepigraph{%
Il y avait, dans cette même contrée des bergers qui passaient dans les champs les veilles de la nuit pour garder leurs troupeaux.}{\ibibleverse{Lc}(2:8)}

Il y a plus de \numprint{2000}~ans, alors que la nuit de Noël s'avançait vers les heures sombres et matinales du jour de Noël, Jésus-Christ est né. Ce qui me frappe vraiment, c'est que les bergers \Og passaient les veilles de la nuit pour garder leur troupeaux \Fg{}, ils veillaient sur leurs troupeaux pendant la nuit!

La partie la plus froide et la plus sombre de la nuit, ce sont les heures du petit matin juste avant l'aube. C'est le moment où des prédateurs nocturnes féroces rôdent dans les champs pour attaquer les agneaux sans défense. Mais ces bergers ne cessaient de veiller sur leurs troupeaux, spécialement au milieu de la nuit. C'étaient de bons bergers.

Quelle belle image cela donne de notre Grand Berger et de ce qu'Il faisait cette nuit-là, il y a si longtemps. Le Seigneur qui veille avec amour sur Son troupeau est né comme un homme qui allait donner Sa vie pour les brebis. Le Bon Berger, Jésus-Christ, demeure avec Ses brebis à tout moment. Même aux moments les plus sombres et les plus froids de la vie, Il est là, \Og passant les veilles de la nuit pour garder Son troupeau. \Fg{} Il veille sur Son troupeau.

Peut-être vous trouvez-vous dans une période de nuit sombre en cette saison de Noël. Le Bon Berger veut veiller sur vous, veiller sur vous pendant votre sombre nuit car vous Lui appartenez.

\begin{dvquotes}
\dvquote{%
Moi, je suis le bon berger.\\
Le bon berger donne sa vie pour ses brebis [\dots]\\
Je connais mes brebis, et mes brebis me connaissent, comme le Père me connaît, et comme je connais le Père; et je donne ma vie pour mes brebis.

J’ai encore d’autres brebis qui ne sont pas de cette bergerie; celles-là, il faut aussi que je les amène; elles entendront ma voix, et il y aura un seul troupeau, un seul berger.

Le Père m’aime, parce que je donne ma vie, afin de la reprendre.}{\ibibleverse{Jn}(10:11,14-17)}
\end{dvquotes}

