\jrnlday{Soleil levant}

\dvepigraph{%
Béni soit le Seigneur, le Dieu d’Israël, de ce qu’il a visité et racheté son peuple, et nous a procuré une pleine délivrance dans la maison de David, son serviteur, comme il en avait parlé par la bouche de ses saints prophètes depuis des siècles, la délivrance de nos ennemis et de la main de tous ceux qui nous haïssent. Ainsi fait-il miséricorde à nos pères et se souvient-il de sa sainte alliance. Selon le serment qu’il a juré à Abraham, notre père. Ainsi nous accorde-t-il, après avoir été délivrés de la main de nos ennemis, de pouvoir sans crainte Lui rendre un culte dans la sainteté et la justice, en sa présence, tout au long de nos jours. Et toi, petit enfant, tu seras appelé prophète du Très-Haut; car tu marcheras devant le Seigneur pour préparer ses voies, pour donner à son peuple la connaissance du salut par le pardon de ses péchés, grâce à l’ardente miséricorde de notre Dieu. C’est par elle que le Soleil Levant nous visitera d’en haut pour éclairer ceux qui sont assis dans les ténèbres et dans l’ombre de la mort et pour diriger nos pas dans le chemin de la paix.
}{\ibibleverse{Lc}(1:68-79) \ocadr Cantique de Zacharie}

Nous voyons dans ce Cantique de Zacharie, qu'il connaissait bien les Écritures et croyaient en leurs promesses. En fait, il s'agit d'une image saisissante de l'ère de l'Ancien Testament cédant la place à l'ère du Nouveau Testament. Dans l'Ancien Testament, seuls certains hommes choisis étaient remplis du Saint-Esprit dans le but d'annoncer la révélation divine. C'est encore vrai ici dans le cas de Zacharie, mais bientôt par l'intermédiaire de Jésus, toute personne qui croirait au Messie allait être remplie du Saint-Esprit.

Le cantique de Zacharie ne parle pas seulement de son fils Jean-Baptiste, il parle du Messie et du salut de Dieu. Il ne comporte que deux versets sur Jean-Baptiste vers la fin. Quand Zacharie commence, "Béni soit le Seigneur, le Dieu d’Israël, de ce qu’il a visité et racheté son peuple," il parle au passé. Ainsi en est-il des prophéties annonçant le futur pour les yeux de Dieu. Si Dieu l'a dit, c'est comme si c'était déjà fait.

Il "nous a procuré une pleine délivrance dans la maison de David". L'expression traduite par "pleine délivrance" est littéralement "corne de délivrance" et fait référence au Messie comme de la corne d'un taureau, symbole de puissance et force.

Jésus le Messie est l'accomplissement de tout l'Ancien Testament. Dans ces versets, Zacharie a presque certainement les noms de sa famille en tête, noms qu'il sait avoir été donnés selon le plan de Dieu: Zacharie – "Le Seigneur se souvient", Élisabeth – "Le serment de Dieu", et Jean – "La Grâce de Dieu".

"Le serment qu’il a juré à Abraham,notre père" après que le patriarche n'ait pas retenu pour lui-même son fils Isaac était la promesse du Messie. Jésus est notre "Soleil Levant", plein de tendre miséricorde, comparable au soleil qui se lève à l'aube avec sa promesse de lumière et d'espoir.

\dvquote{%
Qu’ils sont beaux sur les montagnes, les pieds du messagers de bonnes nouvelles, qui publie la paix! du messager de très bonnes nouvelles, qui publie le salut! qui dit à Sion: Ton Dieu règne!}{\ibibleverse{Es}(52:7)}
        
