\jrnlday{Né SDF}


\dvepigraph{%
Le temps où Élisabeth devait accoucher arriva, et elle enfanta un fils. Ses voisins et ses parents apprirent que le Seigneur avait manifesté envers elle sa miséricorde, et se réjouirent avec elle.}{\ibibleverse{Lc}(1:57-58)}

À l'époque de la première venue de Jésus sur la Terre, la coutûme voulait qu'à la naissance d'un enfant tous les parents et voisins viennent rendre visite à la famille pour se réjouir et célébrer. Et ils faisaient vraiment la fête! Les musiciens du coin venaient avec leurs harpes, leurs lyres, leurs tambourins, et tout le monde chantait des chants d'anniversaire au nouveau-né, le jour même de sa naissance.

Cette célébration a bien eu lieu à la naissance de Jean-Baptiste, mais la naissance de Jésus a été bien différente. Il n'a pas eu de parents ni de voisins pour Lui chanter \emph{Joyeux Anniversaire}. En fait, Marie et Joseph étaient … SDF, \og sans domicile fixe \fg{}, à ce moment-là. Ils logeaient dans une grange.

Réalisez ce que ça signifie … Jésus est né SDF! Il a quitté le trône et le palais céleste et est venu sur terre pour devenir SDF. Plus tard, Il a dit≡ \og Les renards ont des tanières, et les oiseaux du ciel ont des nids; mais le Fils de l’homme n’a pas où reposer sa tête \fg{} (\ibibleverse{Lc}(9:58)). Jésus est passé par tout ça \ocadr et par la croix \fcadr{} pour qu'Il puisse donner à nous qui L'aimons une place au ciel.

\dvquote{Il y a beaucoup de demeures dans la maison de mon Père. Sinon, je vous l’aurais dit; car je vais vous préparer une place. Donc, si je m’en vais et vous prépare une place, je reviendrai et je vous prendrai avec moi, afin que là où je suis, vous y soyez aussi.}{\ibibleverse{Jn}(14:2-3)}
