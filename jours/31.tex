\jrnlday{De solides épaules}

\dvepigraph{%
Car un enfant nous est né, un fils nous est donné, et la souveraineté [reposera] sur son épaule.}{\ibibleverse{Es}(9:5)}

Quand j'étudiais étudié la mythologie grecque au collège, ma divinité mythologique favorite était le titan Atlas, parce qu'il portait le monde sur ses épaules. Les sculptures et les tableaux le représentant le montrent toujours comme un athlète aux muscles énormes; ses muscles trapèzes et deltoïdes sont gigantesques. Bien sûr porter le monde sur vos épaules est un bon exercice pour développer vos deltoïdes!

Atlas devait soutenir le monde parce que lui et les Titans avait perdu une guerre contre Zeus et les dieux de l'Olympe. Zeus a puni Atlas en le condamnant à rester debout et à porter le monde sur ses épaules à perpétuité. Tout ceci, bien sûr, n'est qu'un pur mythe, mais il symbolise bien combien nous pouvons être accablés par des fardeaux comme les soucis, le stress et la peur. Quelques fois nous devons même nous encourager en disant\frcolon{} \og Ne prend pas tous les fardeaux du monde sur tes épaules! \fg{}

Dieu ne veut pas que vous portiez les fardeaux du monde. En fait, la Bible annonce comme une des bénédictions de Noël que nous sommes déchargés de ce fardeau. \og On l’appellera Admirable, Conseiller, Dieu Puissant, Père Éternel, Prince de la Paix \fg{} (\ibibleverse{Es}(9:5)). Jésus se charge de nos besoins, de nos problèmes, de nos soucis et de notre sécurité. Cet Enfant qui nous est né, ce Fils qui nous est donné est le seul capable de porter le monde entier sur Ses épaules.

\dvquote{%
Alors qu'une nouvelle année s'annonce devant vous, pourquoi ne pas faire ce que Jésus vous encourage à faire “Venez à moi, vous tous qui êtes fatigués et chargés, et je vous donnerai du repos.}{\ibibleverse{Mt}(11:28)}

