\jrnlday{Emmanuel --- Dieu avec nous}


\dvepigraph{%
Elle enfantera un fils, et tu lui donneras le nom de Jésus, car c’est lui qui sauvera son peuple de ses péchés… Voici que la vierge sera enceinte; elle enfantera un fils, Et on lui donnera le nom d’Emmanuel, ce qui se traduit\frcolon{} Dieu avec nous.}{%
\ibibleverse{Mt}(1:21,23)}

À Noël, nous célébrons la naissance miraculeuse de Dieu le Fils qui s'est fait homme. Songez-y, Dieu est devenu un homme! Le nom \og Jésus \fg{} est la traduction en grec du nom hébreu \emph{Yeshua} or \emph{Joshua}. Ce nom hébreu de Jésus est en fait la contraction de deux mots\frcolon{} \emph{Jehovah} et \emph{Shua}, qui signifient \og Le Seigneur est notre Sauveur \fg{}. Emmanuel signifie littéralement \og Avec nous est Dieu \fg{}. Au moment de l'incarnation, Dieu est venu pour être avec nous en la personne de Christ, qui est le sacrifice parfait, éternel pour tous nos péchés.

Dans la Création, nous voyons Dieu au-dessus de nous dans sa transcendance. Dans la Loi Morale, nous voyons Dieu comme un juge qui est contre nous. Mais dans les Évangiles, nous voyons Dieu avec nous comme Emmanuel \ocadr notre Sauveur, ami et Roi. En tant qu'Emmanuel, Dieu rejoint Son peuple.

J'aime bien l'histoire du grand-père qui voit son petit-fils sauter dans son parc à bébé en pleurant à chaudes larmes. Quand Petit Pierre voit son grand-père, il lui tend aussitôt ses petites mains potelées en disant\frcolon{} \og Sors-moi, Papy, sors-moi! \fg{}

Tout naturellement, le grand-père se baisse pour soulever Petit Pierre et le faire sortir, mais la maman de l'enfant interrompt\frcolon{} \og Non, Petit Pierre. Tu es puni, tu dois rester dans ton parc! \fg{} Le grand-père, perdu, ne sait plus trop quoi faire. Il est très ému par les larmes de l'enfant. Mais il ne peut prendre à la légère la fermeté de la mère qui veut corriger son enfant. Alors, l'amour trouve une solution. Le grand-père ne pouvant sortir son petit-fils, décide d'enjamber les barreaux pour descendre le rejoindre dans son parc!

C'est précisément ce que notre Seigneur Jésus-Christ a fait pour nous à Noël. En quittant le ciel pour venir sur terre, Il est descendu nous rejoindre!

\dvquote{%
Au commencement était la Parole, et la Parole était avec Dieu, et la Parole était Dieu. […]
 La Parole a été faite chair, et elle a habité parmi nous, pleine de grâce et de vérité;
 et nous avons contemplé sa gloire, une gloire comme celle du Fils unique venu du Père.}{%
\ibibleverse{Jn}(1:1,14)} 

