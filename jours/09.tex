\jrnlday{La maison du pain}


\dvepigraph{%
Et toi, Bethléhem, terre de Juda, tu n’es certes pas la moindre parmi les principales villes de Juda; car de toi sortira un prince, qui fera paître Israël, mon~peuple.}{\ibibleverse{Mt}(2:6)}

Bethléhem, la ville que Dieu a choisie comme lieu de naissance de Son Fils, est à une dizaine de kilomètres au sud de Jérusalem. Elle est située sur un flanc de colline fertile à 750~mètres au-dessus du niveau de la mer. Bethléhem était un endroit parfait pour faire paître des troupeaux. C'est là que David élevait ses moutons et qu'on élevait aussi les moutons destinés aux sacrifices du Temple de Jérusalem. Il était donc ô combien approprié que l'Agneau de Dieu qui allait être sacrifié pour les péchés du monde naquît à Bethléhem!

Le nom \emph{Bethléhem} signifie \Og maison du pain \Fg{}. Il est donc aussi très approprié que Dieu ait choisi ce petit village pour en faire sortir Notre Seigneur et Sauveur Jésus-Christ, Lui qui dirait plus tard\frcolon{} \Og Je suis le pain de vie. \Fg{}

Quand vous y songez bien, les aliments sont très importants pour nos vies. Leurs éléments qui donnent la vie contribuent à notre santé, à notre force et à notre énergie. Ils contribuent aussi à notre plaisir. Certains d'entre nous apprécient probablement trop la nourriture, surtout à Noël, avec les huîtres, la langouste, le saumon, le foie gras, la dinde, les chapons et poulardes, les marrons glacés, les chocolats, les truffes, les bûches de Noël, etc. J'en salive déjà!

La nourriture est vitale pour notre existence et c'est un bon exem\-ple de ce que Jésus est pour nous spirituellement. Il est notre seule source de vie spirituelle, et Il désire nous donner la santé, la force, et l'énergie spirituelles. Sans Jésus nous ne pouvons pas avoir de vie spirituelle ni de plénitude. En cette saison de Noël, Il veut vous donner la vraie joie de connaître et de plaire à Dieu.


%\ornrule

\begin{dvquotes}
\dvquote{%
Car le pain de Dieu, c’est celui qui descend du ciel et qui donne la vie au monde.}{\ibibleverse{Jn}(6:33)}

\dvquote{%
Jésus leur dit\frcolon{}
\Og Moi, je suis le pain de vie.\\
Celui qui vient à moi n’aura jamais faim, et celui qui croit en moi n’aura jamais soif. \Fg{}}{\ibibleverse{Jn}(6:35)}
\end{dvquotes}

