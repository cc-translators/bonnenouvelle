\jrnlday{Une grande lumière est venue}

\dvepigraph[2]{%
Mais les ténèbres ne régneront pas toujours Sur la terre où il y a maintenant des angoisses\frcolon{} si un premier temps a rendu négligeables le pays de Zabulon et le pays de Nephtali, le temps à venir donnera de la gloire à la route de la mer, au-delà du Jourdain, au territoire des nations. Le peuple qui marche dans les ténèbres voit une grande lumière; sur ceux qui habitent le pays de l’ombre de la mort une lumière resplendit.}{\ibibleverse{Es}(9:1-2)}

Les tribus de Zabulon et de Nephtali habitaient la partie nord d'Israël sur les rivages de la Mer de Galilée. C'est une région naturellement belle, mais parce qu'elle se situe à la frange nord de la nation, elle est entourée par des peuples et des nations païennes. Le nom \Og Galilée \Fg{} signifie \Og cercle \Fg{} et la région a toujours été encerclée par des adversaires tels que les Assyriens, les Phéniciens et les Syriens.

Parce que ces nations les envahissaient et prenaient leurs terres régulièrement, Zabulon et Nephtali étaient constamment plongés dans les ténèbres de la guerre et de l'oppression. La Galilée est même devenue connue comme \Og le pays de l'ombre de la mort \Fg{}.

Dieu nous dit à travers la grande prophétie de Noël d'Ésaïe que ces jours de ténèbres sont révolus. Pourquoi? Parce qu'une grande lumière va venir briller sur la Galilée. Cette grande lumière, c'est Jésus. Il a grandi à Nazareth, qui était dans le territoire de Zabulon et puis il a déplacé sa base d'opérations à Capernaüm dans le territoire de Nephtali.

Même \numprint{2000}~ans après que Jésus y a marché, la Galilée continue encore de vibrer de Son passage. Tout \ocadr l'économie, le tourisme et tous les lieux populaires où il a fait la moindre chose \fcadr{} a été pour toujours comme illuminé par Jésus.

Vous arrive-t-il d'avoir le sentiment qu'il y a comme un nuage de ténèbres au-dessus de vous? Qu'une guerre se déroule en vous? Qu'une oppression ou un fardeau continue de vous enfoncer? Quelques fois, Noël peut augmenter notre désarroi parce qu'il se peut que nous ayons des espoirs et des attentes concernant nos familles et nos amis, qui ne se réalisent pas.

Mais il y a une relation qui dépasse toutes les autres et un amour qui va au-delà de toutes les attentes. C'est l'amour que Jésus a pour vous! La présence de Jésus en un endroit efface les ténèbres. Faites lui une place dans votre cœur aujourd'hui.

\begin{dvquotes}
\dvquote{%
En ce jour l’on dira\frcolon{}
\Og Voici notre Dieu, c’est en lui que nous avons espéré et c’est lui qui nous a sauvés;\\
 c’est l’Éternel, en qui nous avons espéré;\\
 soyons dans l’allégresse, et réjouissons-nous de son salut! \Fg{}}{\ibibleverse{Es}(25:9)}
\end{dvquotes}


