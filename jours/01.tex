\jrnlday{Le véritable Père Noël}

Chaque année à Noël, il semble que l'on accorde toujours plus d'attention au Père Noël. Il est la vedette de dessins animés et téléfilms, son portrait se retrouve sur des cartes de vœux, sur du papier cadeau; on peut même le rencontrer \Og en chair et en os \Fg{} dans tous les centres commerciaux. Dans la rue, on peut voir un peu partout son petit mannequin grimper aux immeubles sur une échelle de corde\dots{} Le soir de Noël, on diffuse même des nouvelles de l'approche de son traîneau et de ses rennes!

Pourquoi lui accordons-nous une telle importance? Peut-être est-ce parce que nous voulons tous croire à l'existence d'un être surnaturel qui serait infiniment bon et qui nous montrerait comment vivre et comment nous aimer les uns les autres.

Pour la plupart des gens, le fait qu'il ne s'agisse que d'une fiction n'est pas bien grave; il représente au moins un idéal qui nous fait rêver, chaque année, pendant quelques semaines. Mais en fait, la fiction du Père Noël est basée sur l'existence historique d'une vraie personne!

Il y avait un homme qui s'appelait Nicolas et qui vivait au quatrième siècle dans une région qui correspond à la Turquie actuelle. On l'appelait Saint Nicolas parce qu'il avait vécu une vie chrétienne très pieuse dès le plus jeune âge. On pense que le nom anglo-saxon \emph{Santa Claus} donné au Père Noël vient de \emph{Sinter Klaas}, traduction de \Og Saint Nicolas \Fg{} en hollandais. Saint Nicolas était un homme généreux et l'on se souvient tout particulièrement d'un acte de gentillesse où il avait donné des sacs d'argent à un homme pauvre, père de trois filles. L'argent avait pu servir de dot et permettre aux trois filles de se marier.

Devenu évêque de Myra en Turquie, Saint Nicolas fut persécuté et emprisonné par l'empereur romain Dioclétien en raison de sa grande consécration à Christ. Quand l'empereur Constantin se convertit à la foi chrétienne, Saint Nicolas fut libéré et continua de vivre une vie de service chrétien zélé jusqu'à la fin de ses jours.

Saint Nicolas était un homme rempli de l'esprit de joie et de générosité, non pas parce qu'il croyait à un mythe, mais parce qu'il croyait au Divin Sauveur. Si le Père Noël, joyeux, corpulent, vêtu d'un costume rouge, domicilié au Pôle Nord, est une légende, il existe bien en revanche une véritable personne surnaturelle en qui vous pouvez croire et sur qui vous pouvez compter tous les jours de l'année. Cette personne est Jésus-Christ.

\begin{dvquotes}
  \dvquote{%
On l'appellera Admirable, Conseiller, Dieu Puissant, Père éternel, Prince de Paix%
}{\ibibleverse{Es}(9:5)}
\end{dvquotes}
