\jrnlday{Le roi de c\oe{}ur}

\dvepigraph{%
On l’appellera Admirable, Con\-seil\-ler, Dieu Puissant, Père Éternel, Prince de la Paix.}{\ibibleverse{Es}(9:5)}

Jésus est né pour prendre soin et régner sur les cœurs des hommes. Il est le Roi de cœur, et Il est le Roi des rois et le Seigneur des seigneurs. Il dispose d'une sagesse, d'une puissance, et d'un amour infinis pour gouverner Son royaume et Ses sujets. En fait quand vous entrez dans Son royaume qui est constitué par tous ceux qui Lui ont remis leurs vies, Son règne de paix continue de grandir et de s'étendre à travers vous. Le Seigneur est plein de zèle pour le soin et le bien-être de Son peuple.

Jésus est appelé Admirable. C'est un nom en hébreu, pas un adjectif, et il signifie \Og celui qui accomplit des merveilles \Fg{} ou \Og homme miracle \Fg{}. Le plus grand miracle de Jésus, c'est qu'Il transforme nos vies. Il change nos espérances et nos attitudes qui de centrés sur nous-mêmes deviennent centrés sur Dieu. Jésus est la suprême Merveille du Monde, et Noël en est la preuve.

Jésus est appelé Conseiller. Le Conseil de Sa Parole et de Son Esprit est le seul conseil qui peut vous libérer de tout complexe et de tout esclavage.

Jésus est appelé Dieu Puissant \ocadr \emph{El Gibbôr} en hébreu \fcadr{} qui signifie \Og Dieu de Puissance \Fg{} ou \Og Dieu le Puissant Champion \Fg{}. Cet enfant était Dieu Lui-même et Il a conquis la mort pour nous racheter. Il est le Champion du Monde.

Jésus est appelé Père Éternel, littéralement \Og celui qui est à l'origine de l'éternité \Fg{}. C'est pourquoi Il peut nous offrir le don de la vie éternelle, que l'on trouve en le connaissant parce qu'Il est la vie éternelle.

Jésus est appelé Prince de la Paix ou \Og dirigeant de paix \Fg{}, ou \Og celui qui apaise \Fg{}. Le règne de Jésus dans nos cœurs est le seul moyen de connaître la vraie paix de l'esprit. C'est le gouvernement de Jésus-Christ, et vous pouvez avoir ce règne de paix et d'amour aujourd'hui.

\dvquote{%
Renforcer la souveraineté et donner une paix sans fin […]\\
Voilà ce que fera le zèle de l’Éternel des armées.
}{\ibibleverse{Es}(9:6)} 






