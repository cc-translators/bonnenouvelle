\jrnlday{Le saint enfant}

\dvepigraph[2]{%
Au sixième mois, l’ange Gabriel fut envoyé par Dieu dans une ville de Galilée du nom de Nazareth, chez une vierge fiancée à un homme du nom de Joseph, de la maison de David; le nom de la vierge était Marie. Il entra chez elle et dit\frcolon{} Je te salue toi à qui une grâce a été faite; le Seigneur est avec toi.

Troublée par cette parole, elle se demandait ce que signifiait une telle salutation. L’ange lui dit\frcolon{} Sois sans crainte, Marie; car tu as trouvé grâce auprès de Dieu. Voici\frcolon{} tu deviendras enceinte, tu enfanteras un fils, et tu l’appelleras du nom de Jésus. Il sera grand et sera appelé Fils du Très-Haut, et le Seigneur Dieu lui donnera le trône de David, son père. Il règnera sur la maison de Jacob éternellement et son règne n’aura pas~de~fin.

Marie dit à l’ange\frcolon{} Comment cela se produira-t-il, puisque je ne connais pas d’homme? L’ange lui répondit\frcolon{} Le Saint-Esprit viendra sur toi, et la puissance du Très-Haut te couvrira de son ombre. C’est pourquoi, le saint (enfant) qui naîtra sera appelé Fils de Dieu. Voici qu’Élisabeth ta parente a conçu, elle aussi, un fils en sa vieillesse, et celle qui était appelée stérile est dans son sixième mois. Car rien n’est impossible à Dieu.

Marie dit\frcolon{} Voici la servante du Seigneur; qu’il me soit fait selon ta parole. Et l’ange s’éloigna d’elle.%
}{\ibibleverse{Lc}(1:26-38)}

Après que l'ange Gabriel fut apparu à Zacharie à Jérusalem pour lui annoncer la naissance de son fils, le futur Jean-Baptiste, il fut envoyé à 110~km au nord-est, à Nazareth, une ville de près de \numprint{15000}~habitants. L'ange était envoyé à Marie, une jeune vierge fiancée à Joseph, pour faire l'annonce d'une naissance encore bien plus spectaculaire.

Plutôt que de réagir par le doute comme l'avait fait Zacharie peu avant, Marie répond à Gabriel avec une foi pleine d'humilité et un esprit de soumission. Sans doute est-ce pour cela que Marie avait été choisie pour un tel honneur. La Bible ne dit pas qu'elle fut troublée par l'ange lui-même mais par son message. Peut-être se sentait-elle indigne d'une telle salutation. Étant encore vierge, elle avait des questions légitimes sur la façon dont elle tomberait enceinte, pas sur le fait que ça arriverait.

La naissance de Jésus fut unique. Il fut le seul homme engendré physiquement par un acte direct de Dieu. L'Esprit Saint est venu sur Marie et la puissance du Très-Haut l'a couverte de son ombre. De cette façon, Jésus était complètement homme par Marie et complètement Dieu par cet acte direct de Dieu. Cette puissance du Très-Haut qui a couvert Marie de son ombre a permis de garder l'enfant ainsi conçu pur et non souillé par la nature pécheresse de Marie fille de l'homme déchu. Jésus est appelé le \Og Saint Enfant \Fg, un homme parfaitement sans péché. La vérité de la naissance d'une vierge est une doctrine chrétienne cruciale, parce que si Jésus n'était pas Dieu et parfaitement sans péché, Il ne pourrait pas être le Sauveur qui est mort pour les péchés du monde.


\begin{dvquotes}
  \dvquote{%
    Le lendemain, Jean vit Jésus venir à lui et dit\frcolon{} Voici l’Agneau de Dieu, qui ôte le péché du monde.
  }{\ibibleverse{Jn}(1:29)}
\end{dvquotes}
