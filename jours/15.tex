\jrnlday{Des langes}

\dvepigraph{%
Et elle enfanta son fils premier-né. Elle l'emmaillota et le coucha dans une crèche, parce qu'il n'y avait pas de place pour eux dans l'hôtellerie.}{\ibibleverse{Lc}(2:7)}

Quand Jésus est né, Marie l'a emmailloté dans des langes. L'emmaillotage était une coutume orientale qui consistait à serrer le\linebreak nouveau-né dans des bandes de linge, en tenant les bras près du corps pour lui donner un sentiment de sécurité et réconfort.

Luc était un médecin et le mot grec qu'il utilise pour désigner les langes et le terme médical qui désigne des "bandages". Ici nous avons un premier indice nous indiquant pourquoi Jésus est venu dans un monde qui a de la peine et génère de la peine – Il est venu pour guérir. Pas seulement pour mettre un pansement sur nos blessures, mais pour porter dans Son corps toute notre peine et la cause même de notre peine, notre péché.

Peut-être ressentez-vous de la douleur aujourd'hui; On a commis des péchés terribles envers vous ou vous vous êtes vous-même infligé de la peine, ou vous ressentez simplement la douleur qu'il y a à vivre sur cette planète déchue. Jésus sait ce que vous traversez. Vous pouvez venir auprès du Grand Médecin et le laisser vous soigner.

La même pratique d'emmaillotage des bébés était utilisée pour habiller le corps d'une personne morte avec ses vêtements funéraires. Il y avait donc le petit Jésus, un bébé emmaillotté dans des langes; Le cher Jésus, né pour en fin de compte mourir pour nos péchés, emmaillotté dans des tissus de Sa naissance à Sa mort. Mais ces vêtements funéraires n'allaient pas pouvoir retenir Jésus quand Il est ressuscité et a conquis la mort!


\begin{dvquotes}
\dvquote{%
Simon Pierre qui le suivait, arriva. Il entra dans le tombeau, aperçut les bandelettes qui étaient là… Alors l’autre disciple, qui était arrivé le premier au tombeau, entra aussi; il vit et il crut. Car ils n’avaient pas encore compris l’Écriture, selon laquelle Jésus devait ressusciter d’entre les morts}{\ibibleverse{Jn}(20:6,8-10)}
\end{dvquotes}
