\jrnlday{L'ouvrage de Dieu}

\dvepigraph{%
Car nous sommes Son ouvrage, nous avons été créés en Christ-Jésus pour des œuvres bonnes que Dieu a préparées d'avance, afin que nous les pratiquions.}{\ibibleverse{Ep}(2:10)}

Pendant la Renaissance, il y avait un sculpteur du nom d'Agostino d'Antonio qui vivait à Florence en Italie. Il travaillait un énorme bloc de marbre, mais n'aboutissait à rien. Finalement dans sa frustration, il se dit "Je ne peux rien faire avec ce morceau de marbre minable" et il l'abandonna dans un coin. D'autres sculpteurs essayèrent d'en tirer quelque chose, mais ils abandonnèrent aussi. Le morceau de pierre fut mis au rebut et resta oublié dans un coin pendant 40 ans.

Puis un jour, un sculpteur du nom de Michel-Ange qui passait par-là, tomba par hasard sur l'endroit et remarqua le morceau de marbre. Il aima ce qu'il vit. On transporta le marbre dans son atelier et il se mit au travail, le sculptant et le transformant avec patience et amour. Finalement, la pierre qui avait semblé n'avoir aucune valeur devint       "David" le plus grand chef-d'œuvre de Michel-Ange qui, incroyablement, frémit de vie bien qu'il soit beaucoup plus grand que nature.

Plus tard dans sa vie, Michel-Ange qui venait de terminer la sculpture qui est considérée comme sa plus grande œuvre - Moïse et les Dix Commandements – est tombé dans une rage de dépit. Il lança un burin contre la sculpture, causant un éclat dans le genou de Moïse qui est encore visible aujourd'hui. (Apparemment, Moïse avait un problème en matière de "gestion de la colère"!). Dieu merci, avant de mourir, Michel-Ange a réalisé combien il avait besoin de Jésus et il a écrit dans son testament "Je meurs dans la foi en Jésus-Christ et dans la ferme espérance d'une vie meilleure."

Nous essayons si fort de trouver la plénitude en nous-mêmes, mais il nous manque toujours quelque chose. Il y a toujours un manque, une solitude une imperfection. Nous essayons de remplir ce vide avec des plaisirs, des relations, des succès ou des vices. Nous tentons de trouver la satisfaction dans l'alcool, les drogues, le jeu, les sports, la musique, les films. Mais le vide subsiste. C'est pour ça que nous avons tant besoin de Jésus.

Les anges l'ont dit parfaitement aux bergers:  "Aujourd'hui, dans la ville de David, il vous est né un Sauveur, qui est le Christ, le Seigneur" (Luc 2:11). Nous avons tous besoin d'un Sauveur bien plus que nous pouvons même l'imaginer. "Je dis à l'Éternel: Tu es mon Seigneur, Mon Bien, Il n'y a rien au-dessus de toi!" (Psaume 16:2).

