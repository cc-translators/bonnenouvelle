\jrnlday{Laissons Dieu faire Son travail}


\dvepigraph{%
Voici comment arriva la naissance de Jésus-Christ. Marie, sa mère, était fiancée à Joseph; avant leur union elle se trouva enceinte (par l’action) du Saint-Esprit.}{\ibibleverse{Mt}(1:18)}

Chez les Hébreux, les fiançailles étaient bien plus qu'un simple engagement. Légalement, un couple fiancé était comme marié et consacrait sa première année à cultiver l'amitié et la consécration réciproque, sans vivre ensemble ni avoir de relations sexuelles. Dans la société d'aujourd'hui, les couples font tout l'opposé – ils se précipitent dans des relations sexuelles sans mariage et ne cultivent pas l'engagement d'un amour vrai. En conséquence, les relations sont souvent égoïstes, superficielles et douloureuses.

Pendant la période des fiançailles de Joseph et Marie, Joseph a appris que Marie était enceinte. Nous ne savons pas si Marie a essayé d'expliquer à Joseph l'apparition de l'ange Gabriel et la conception miraculeuse. Il est très possible que Marie n'ait même pas essayé de lui en parler. Vraiment, qu'allait-elle lui expliquer? "Joseph, mon chéri, Je suis enceinte, mais ce n'est pas ce que tu crois. Ce n'était pas un homme, c'était le Saint-Esprit." Ça pouvait être dur à avaler!

Il se pourrait que Marie ait simplement remis sa situation entre les mains de Dieu. Peut-être s'est-elle dit, Ce n'était pas mon idée, c'était celle du Seigneur. Il faudra qu'Il règle le problème car moi je ne le peux pas. Et Le Seigneur a réglé le problème en inspirant un rêve à Joseph.

Bien des fois, nous nous trouvons incapables de convaincre les gens ou de les faire changer d'avis. Nous savons seulement que ce que Dieu dit, Il a l'intention de l'accomplir. Nous devons remettre ces gens à Dieu et croire qu'Il va les convaincre et les transformer. C'est Sa tâche. Nous sommes appelés à prier et à aimer; la part de Dieu c'est d'accomplir Sa volonté dans la vie des gens. Et Il le fera fidèlement.


\dvquote{%
Confie-toi en l’Éternel de tout ton c\oe{}ur, et ne t’appuie pas sur ton intelligence;
 Reconnais-le dans toutes tes voies, et c’est lui qui aplanira tes sentiers.}{\ibibleverse{Pr}(3:5-6)} 


                      

